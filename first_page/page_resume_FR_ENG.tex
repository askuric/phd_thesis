\documentclass[french,12pt,a4paper]{report}
\usepackage[top=0.8cm,bottom=0.8cm,left=1.5cm,right=1.5cm]{geometry}
\usepackage[T1]{fontenc}
\usepackage[utf8]{inputenc}
\usepackage{setspace}
\usepackage{graphicx}
\usepackage[french,english]{babel}
\usepackage{microtype} % diminue les hyphénations trop fréquentes

\begin{document}
\thispagestyle{empty}
\vspace*{0pt}
\vfill
\begin{small}
\begin{center}
\textbf{Une vue couplée des capacités physiques de la dyade humain-robot pour l'évaluation quantitative en ligne des besoins d'assistance}
\end{center}    
\textbf{Résumé :} 
L'objectif de ce travail est de développer une vision unifiée des différentes mesures de capacité physique pour les robots, les humains et leurs capacités communes lorsqu'ils collaborent, dans le but d'améliorer les performances et la qualité de leur collaboration. La famille de mesures sur laquelle repose ce travail est celle des mesures de capacité physique exprimées sous forme de polytopes convexes. Les polytopes convexes sont particulièrement adaptés aux scénarios de collaboration car ils peuvent être facilement visualisés (représentés sous la forme d'un maillage triangulé de sommets) et intégrés aisément dans différents problèmes d'optimisation (représentés comme un ensemble de contraintes linéaires). De plus, la possibilité d'effectuer efficacement des opérations sur les polytopes (telles que la somme, l'intersection et l'union), en utilisant l'algèbre des polytopes, permet de caractériser les capacités physiques de plusieurs robots et humains dans des scénarios de collaboration.

Deux algorithmes capables de calculer en temps réel le polytope de force des manipulateurs robotiques et des modèles musculo-squelettiques humains ont été développés. Le potentiel de ces mesures en temps réel pour être utilisées dans des stratégies de commande flexible des robots dans des scénarios de collaboration (robot-robot et humain-robot) est démontré dans le contexte du transport collaboratif d'objets. Enfin, ces mesures sont utilisées pour quantifier les capacités physiques communes dans les scénarios de collaboration.

Une nouvelle mesure, sous la forme d'un polytope convexe, est introduite pour l'approximation de l'espace de travail atteignable du manipulateur robotique. Cette mesure permet de calculer l'espace de travail atteignable du robot dans un horizon temporel d'intérêt et tient compte de la dynamique du robot, de sa charge utile et de ses limites d'actionnement.

Enfin, une approche de planification de trajectoire dans l'espace cartésien exploitant la capacité de mouvement du robot basée sur les polytopes est présentée. Cette approche utilise l'algèbre des polytopes pour évaluer efficacement la capacité de mouvement du robot dans la direction de la trajectoire en temps réel et exploite pleinement sa capacité de mouvement en mettant à jour la trajectoire planifiée en temps réel.\\
\textbf{Mots-clés :} interaction humain robot,robotique collaborative, analyse des performances, géométrie computationnelle, commande\\
%\noindent\rule{\textwidth}{1pt}
\noindent\makebox[\linewidth]{\rule{\textwidth}{0.4pt}}

\selectlanguage{english}
\begin{center}
\textbf{A coupled view of the physical abilities of human-robot dyad for theonline quantitative evaluation of assistance needs}
\end{center}
\textbf{Abstract:} 
The goal of this work is to develop an unifying view of different physical ability metrics for robots, humans and their common abilities when collaborating, with the aim to improve the performance and the quality of their collaboration. The family of metrics this work is based on are the physical ability metrics expressed in a form of convex polytopes. Convex polytopes are particularly suitable for collaborative scenarios as they can be easily visualised (represented as a triangulated mesh of vertices) and can be easily integrated in different optimisation problems (represented as a set of linear constraints). Additionally, being able to do operations over polytopes efficiently (such as sum, intersection and unions), using the polytope algebra, enables characterisation of physical abilities of multiple robots and humans in collaborative scenarios.

Two algorithms capable of calculating the real-time force polytope of robotic manipulators and human's based on the musculoskeltal models are developed. The potential of these real-time metrics to be used for flexible robot control strategies in collaborative scenarios (robot-robot and human-robot) is demonstrated within the collaborative object carrying setting. Finally, these metrics are used to quantify the common physical abilities in the collaboration scenarios. 

A new metric for the approximation of the robotic manipulator's reachable workspace in a form of a convex polytope is introduced. This metric calculates the reachable workspace of the robot within a time horizon of interest and is capable of taking in consideration the robot's dynamics, payload and its actuation limits. 

Finally, a cartesian space trajectory planning approach that exploits robot's polytope based motion capacity is introduced. This approach leverages the polytope algebra to efficiently evaluate the robot's movement capacity in the trajectory direction online and exploits the robot's full movement capacity by updating the planned trajectory in real-time.\\
\textbf{Keywords:} human robot interaction, collaborative robotics, performance analysis, computational geometry, control \\
\noindent\makebox[\linewidth]{\rule{\textwidth}{0.4pt}}

\vfill
\selectlanguage{french}
\begin{center}
    \textbf{INRIA de l'Université de Bordeaux}\\
%UMR xxxx 
Université, 33000 Bordeaux, France.
\end{center}
\end{small}
\vfill
\end{document}
