% Chapter 1

\chapter{Introduction} % Main chapter title

\label{ch:intro} % For referencing the chapter elsewhere, use \ref{Chapter1} 


% \todos{
% Here the main goal is to introduce why using physical ability metrics could be beneficial for human-robot collaboration and pose a set of questions that we will try to reply in this thesis
% }

% Current plan
% \url{https://notes.inria.fr/gY2-Ip7_TqqJfXr9ncDYZA}

The industry of the future, as described by the movement called Industry 5.0~\cite{MADDIKUNTA2022ind50}, promises to step away from the traditional point of view on industrial automation, where the goal is to blindly automatise, as much processes as possible, in order to improve the overall efficiency and reduce the cost. The Industry 5.0 puts the human workers in the central position and aims to create the workflows that ensure their well-being as well as the long term sustainability of the industrial practices in general~\cite{XU2021ind50}. The industry of the future relies on the flexibility and adaptability of the human workers, by embracing their cognitive and physical skills as well as their talents and different levels of expertise. The role of the automation is no longer purely optimisation of the industrial processes, but providing the support and assistance to the human operator with the aim to exploit both machines' and humans' full potential, establishing human-automation symbiosis~\cite{LENG2022ind50} on the industry floor.
 
One typical example of such symbiosis is an industrial workstation~\cite{SIMOES2022workplace}, where the human and the robot work in a close proximity and interact physically to execute different tasks. The aim of such collaboration is to improve the overall efficiency by exploiting the abilities of both humans (flexibility, adaptability, cognitive capacity, expertise, etc.) and robots (repeatability, precision, tirelessness, etc.), as well as to improve the operators well-being, remove the unnecessary strain and ensure his safety when executing tasks. Therefore, such future collaborative scenarios require having a set of tools for characterising different abilities of robots and humans, as well as different notions of human well-being and safety. Such tools are required both for assessing different tasks and allocating them with respect to humans and robots abilities as well as for creating more human-centred robot control strategies.

%\todos{I'm more or less happy till here}

Even though human-robot collaboration is still relatively recent field, there are numerous performance metrics proposed in the literature that quantify different aspects of the quality of their collaboration and their individual abilities~\cite{CORONADO2022collab_quality}.
The main focus of this work is put on different tools for characterising human's and robot's physical abilities, especially their polytope representations, which have a great potential to be used in this context. Both in order to quantify and leverage their individual abilities as well as to ensure human safety. As discussed in \Cref{ch:poly_metrics}, the polytope characterisations of different physical abilities are an accurate representations of physical abilities for both robots and humans. Furthermore, the polytopes enable expressing their individual physical abilities, as well as their joint physical abilities when collaborating, in the same polytope form. Such unified view on their capacities lays foundation for creating new task allocation strategies that take in consideration their physical abilities, by being able to asses if different tasks better suite human's or robot's abilities or potentially require their collaboration. 

Furthermore, both human's and robot's physical abilities are state dependant and can vary significantly during the task execution, and polytopes enable capturing the changes in their abilities as well. Having an accurate information about the robot's physical abilities online, enables creating robot control strategies that adapt to the their changes and exploit the robot's full potential. On the other hand, having an accurate online information about the human's physical abilities is important for assessing if the operator is apt to execute a certain task and quantifying if the operator needs assistance of the robot due to its lacking physical abilities. Furthermore, having the real-time information about the operator's physical abilities enables ensuring that his abilities are never surpassed which has a direct impact on the operator's safety and well-being. Therefore, the collaborative workstations require creating more human-centred robot control strategies where the robot adapts its behaviour not just with respect to the requirements of the task and its own abilities, but also to the current abilities of the human operator as well as his safety. As polytope characterisations of robot's and human's physical abilities can be easily integrated with different robot control strategies, they have a great potential to be used for creating such adaptive collaboration scenarios.
